\documentclass[../main.tex]{subfiles}

\begin{document}
\section{Decisioni di breve periodo}

\subsection{Margine di contribuzione}
Nel breve periodo \textbf{i costi fissi sono tipicamente non evitabili}. Assume quindi un ruolo rilevante il margine di contribuzione.

\begin{itemize}
    \item \textbf{Margine di contribuzione unitario}: differenza tra prezzo di vendita e costo variabile unitario: $\mathbf{m = p - cv}$
    \item \textbf{Margine di contribuzione totale}: $\mathbf{M = m \cdot Q}$ (Q: quantità prodotta)
    \item \textbf{Margine di contribuzione medio}: Si definisce nel caso di impresa multi-prodotto. Si fa una media pesata dei margini di contribuzione a seconda della quantità prodotta.
\end{itemize}

\subsection{Tipologie di decisioni di breve periodo}

\subsubsection{Make or buy}
Decisioni inerenti la scelta tra: 
\begin{itemize}
    \item produrre un determinato input/componente/prodotto all'interno dell'azienda (MAKE)
    \item acquistare l'input/componente/prodotto sul mercato (BUY)
\end{itemize}

I passi della scelta:
\begin{itemize}
    \item si identificano le alternative di make or buy
    \item si adotta una delle due alternative come caso base
    \item si calcolano costi e ricavi differenziali rispetto al caso base
    \item si preferisce l'alternativa che crea maggiore valore
\end{itemize}

Nelle scelte di make or buy è necessario anche considerare \textbf{i costi opportunità}: Beneficio al quale si rinuncia quando una determinata scelta implica l'esclusione di ...

Possono esistere anche scelte di \textbf{make or buy di lungo periodo}.

Le scelte di make or buy prescindono da considerazioni di tipo qualitativo:
\begin{itemize}
    \item qualità del lavoro del fornitore
    \item affidabilità del fornitore in termini di puntualità delle consegne
    \item eventuale stagionalità del fabbisogno di componenti
    \item livello di riservatezza delle conoscenze necessarie a produrre un componente
\end{itemize}
Non tengono conto dei costi di transazione (costi di organizzazione e gestione degli scambi).

\subsection{Analisi di break-even}

Valutazioni relative a quanto è necessario produrre per coprire i costi (caso 1) o per ottenere un certo profitto target (caso 2) a risorse date.

$$\mathbf{Q_{BE}: ricavi\ totali - costi\ totali = 0}$$

$$\mathbf{Q_{target}: ricavi\ totali - costi\ totali = profitto\ target}$$

\subsubsection{Ipotesi semplificatrici}
\begin{enumerate}
    \item Ipotesi sui \textbf{costi}: costi variabili unitari costanti, non cambiano al variare del volume produttivo
    \item Ipotesi sui \textbf{ricavi}
        \begin{itemize}
            \item I ricavi sono realizzati immediatamente
            \item Non vi sono scorte invendute
        \end{itemize}
    \item Ipotesi sul \textbf{prezzo}: costante rispetto al volume di vendita: non cambia al variare del volume produttivo
\end{enumerate}

\subsubsection{Caso 1 (imprese monoprodotto)}
$$\mathbf{p \cdot Q_{BE} - cv \cdot Q_{BE} - CF = 0}$$
$$\mathbf{???}$$

\subsubsection{Caso 2 (imprese monoprodotto)}
$$\mathbf{p \cdot Q_{target} - cv \cdot Q_{target} - CF = target}$$
$$\mathbf{???}$$

\subsubsection{Imprese multi-prodotto}
Nel caso multiprodotto, si suppone che il mix produttivo sia definito da percentuali prefissate ($x_j$) di N prodotti, andando quindi a definire un margine di contribuzione medio
$$\mathbf{m_{medio} = \sum_{j=1}m_j \cdot x_j}$$

\subsubsection{Interpretazione della quantità di break-even}

La quantità di break-even:
\begin{itemize}
    \item Indica il minimo numero di prodotti da vendere per avere un profitto non negativo
    \item Consente all'impresa di valutare il proprio \textbf{margine di sicurezza}
\end{itemize}

$\mathbf{Margine\ di\ sicurezza = ???}$

\subsubsection{Considerazioni conclusive}
In caso di alti costi operativi fissi, la genstione di impresa è sottoposta a rischi elevati

$$\mathbf{Indice\ di\ rigidita = \frac{costi\ fissi}{costi\ totali}}$$

In caso di elevata rigidità, la quantità di BE aumenta
\begin{itemize}
    \item per avere un profitto positivo sono necessari elevati volumi di produzione e vendità
    \item Shock negativi di domanda possono causare pesanti perdite
\end{itemize}

Strategie di variabilizzazione dei costi: trasformare i costi fissi in costi variabili
\begin{itemize}
    \item Ricorso all'outsourcing di servizi
    \item Acquisto esterno di componenti e semilavorati
\end{itemize}


\subsection{Scelta del mix produttivo}

Quanto produrre di ogni prodotto, nel caso di azienda multi-prodotto?

\begin{itemize}
    \item Quale prodotto è più opportuno realizzare
    \item Quanto conviene produrre di ciascuno dei prodotti dell'impresa qualora esistano
        \begin{itemize}
            \item vincoli relativi al consumo di risorse
            \item vincoli contrattuali
            \item vincoli di mercato
        \end{itemize}
\end{itemize}

Gli step della scelta
\begin{enumerate}
    \item Si calcola il margine di contribuzione di ciascun prodotto e si verifica che sia positivo (\textbf{i prodotti con margine negativo non devono essere considerati})
    \item Si prendono in esame i vincoli:
        \begin{enumerate}
            \item In \textbf{assenza di vincoli}: si produce il prodotto con margine di contribuzione maggiore
            \item In presenza di \textbf{vincoli di consumo di risorse}: si massimizza il margine di contribuzione per risorsa scarsa
            \item In presenza di \textbf{vincoli contrattuali}: si soddisfano gli eventuali vincoli contrattuali e si massimizza il margine di contribuzione (o il margine di contribuzione per risorsa scarsa)
            \item In presenza di \textbf{vincoli di mercato}: si massimizza il margine di contribuzione (o il margine di contribuzione per risorsa scarsa) tenendo conto del limite superiore imposto dalla domanda
        \end{enumerate}
\end{enumerate}
\end{document}