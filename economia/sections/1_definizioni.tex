\documentclass[../main.tex]{subfiles}

\begin{document}
\section{Definizioni Giuridiche}

\subsection{Attività}

\begin{itemize}
    \item \textbf{Economica:}

          l'output deve poter essere oggetto di \textbf{scambio} su un mercato (deve avere un valore \textbf{economico}).

    \item \textbf{Professionale:}

          svolta abitualmente, ma non necessariamente:
          \begin{itemize}
              \item con continuità temporale
              \item in esclusiva
              \item dall'imprenditore (è possibile delegare)
          \end{itemize}

    \item \textbf{Organizzata:}

          l'impresa ha una sua organizzazione, struttura che consente una \textbf{gestione coordinata delle risorse}.\\
          l'imprenditore organizza liberamente l'impresa.

\end{itemize}

\subsection{Impresa vs Società vs Azienda vs Ditta}

Le imprese possono essere società, ma non necessariamente\\
Azienda e ditta hanno altri significati rispetto ad impresa

\begin{itemize}
    \item \textbf{Società:}\\
          contratto con cui due o più persone conferiscono beni o servizi per l'esercizio in comune di una attività economica allo scopo di dividerne gli utili

    \item \textbf{Azienda:}\\
          complesso dei beni organizzati dall'imprenditore per l'esercizio dell'impresa.

    \item \textbf{Ditta:}\\
          nome commerciale scelto dall'imprenditore per esercitare l'impresa.\\
          ha valore commerciale (google, apple, ferrari), per questo, la legge ne garantisce \textbf{l'uso esclusivo}.

\end{itemize}
\end{document}