\documentclass[../main.tex]{subfiles}

\begin{document}
\section{Forme Giuridiche}

La forma giuridica definisce quali sono i diritti e doveri di una impresa.

Esempi di \textbf{diritti di una impresa}:
\begin{itemize}
	\item Diritti di proprietà su beni e servizi utilizzati per l'esercizio dell'attività.
	\item Diritto di affittare un immobile (essere intestataria del contratto di affitto)
	\item Diritto a difendersi in tribunale in caso di controversie.
\end{itemize}

Esempi di \textbf{doveri di una impresa}:
\begin{itemize}
	\item Dovere di redigere il bilancio.
	\item Dovere di rispettare le leggi a tutela dei lavoratori
	\item Dovere di rispettare la normativa ambientale.
\end{itemize}

Per \textbf{forma giuridica} si intende la \textbf{tipologia giuridica del soggetto a cui fa capo l'attività} e le norme ad essa conseguenti.

La scelta della forma giuridica è importante perchè determina una serie di \textbf{obblighi civili, amministrativi e fiscali dell'impresa}.

Il codice civile distingue:

\begin{enumerate}
	\item \textbf{Imprese individuali:} costituite  da un'unica persona fisica. Non c'è distinzione giuridica tra il proprietario e l'impresa stessa.

	\item \textbf{Imprese collettive:} società di più persone.
\end{enumerate}

\subsection{Imprese individuali}

\textbf{Il titolare è illimitatamente responsabile delle obbligazioni} dell'impresa con tutto il patrimonio personale.

Tipica di attività quali: commercialista, architetto, ingegnere, medico, consulente di vario genere.

\textbf{Impresa familiare:} estensione dell'impresa individuale, quando l'imprenditore si avvale in modo continuativo della prestazione lavorativa dei familiari. \\

\textbf{PRO:}
\begin{itemize}
	\item \textbf{Semplicità} nella costituzione e lo scioglimento dell'impresa. Non è richiesto il versamento del capitale.

	\item \textbf{Pochi obblighi contabili}, non sono obbligate a redigere un bilancio in forma complessa.

	\item \textbf{Autonomia e velocità} decisionale.
\end{itemize}

\textbf{CONTRO:}
\begin{itemize}
	\item \textbf{Responsibilità illimitata:} l'imprenditore risponde con il proprio patrimonio personale per le obbligazioni assunte nel corso dell'attività.

	\item In caso di forti guadagni le \textbf{imposte} crescono (si applicano aliquote progressive previste dall'Irpef).

\end{itemize}

\subsection{Imprese collettive}

Le imprese collettive si distinguono principalmente in:

\begin{itemize}
	\item \textbf{Società di persone}
	      Soci hanno responsabilità solidale e illimitata per le obbligazioni sociali (con alcune eccezioni). In queste società i soci sono tassati come se fossero individui.

	\item \textbf{Società di capitale}
	      Soci hanno responsabilità limitata e circoscritta ai loro rispettivi conferimenti sociali.
	      C'è distinzione giuridica tra proprietario e impresa.

	\item \textbf{Società di cooperative} Soci hanno responsabilità limitata e circoscritta ai loro rispettivi conferimenti sociali.
	      Si contraddistinguono per lo scopo mutualistico.
\end{itemize}

\subsubsection{Società di persone}

\begin{itemize}
	\item \textbf{Società semplice (s.s.):} riservata ad attività economiche non commerciali (attività agricole e per la gestione di patrimoni immobiliari)
	\item \textbf{Società in nome collettivo (s.n.c.):} può esercitare sia attività di impresa commerciale, sia attività economiche non commerciali.
	\item \textbf{Società in accomandita semplice (s.a.s.)}
	      \begin{itemize}
		      \item \textbf{soci accomandatari:} si assumono in forma illimitata e solidale le responsabilità connesse all'esercizio dell'impresa.
		      \item \textbf{soci accomandanti:} affidano in gestione i loro capitali ad altri soci e sono responsabili sono del capitale conferito. Per questi soci vale la responsabilità limitata.

	      \end{itemize}
\end{itemize}

\textbf{PRO:}
\begin{itemize}
	\item Costituzione e la tenuta della contabilità relativamente semplici.

	\item \textbf{Procedure} burocratiche, fiscali, contabili e tributarie minime.

	\item \textbf{Non è obbligatorio il versamento di un capitale minimo} da parte dei soci (l'importo è stabilito dal contratto sociale).

	\item Più facile espandersi e trovare capitale addizionale.

\end{itemize}

\textbf{CONTRO:}
\begin{itemize}
	\item \textbf{Responsabilità illimitata} (a parte accomandanti della s.a.s.) e \textbf{solidale}: in caso di fallimento, i creditori possono rivalersi sul patrimonio privato di ciascun membro della società. \textbf{Se un socio non adempie, il debito dovrà essere saldato dagli altri!}
	\item Minore autonomia decisionale, problemi di \textbf{coordinamento}.

\end{itemize}

\subsubsection{Società di capitali}

\begin{itemize}
	\item \textbf{Società a responsabilità limitata (s.r.l.):} il capitale sociale (la proprietà) è diviso in \textbf{quote} (le quote non possono essere oggetto di sollecitazione all'investimento del pubblico risparmio). C'è un capitale minimo di 10 000 \euro.

	\item \textbf{Società a responsabilità limitata semplificata (s.r.l.s.):} forma di s.r.l. recentemente introdotta per favorire l'imprenditorialità. Capitale minimo di 1 \euro.

	\item \textbf{Società per azioni (s.p.a.):}

	      \begin{itemize}
		      \item
		            Il pratimonio sociale è costituito da \textbf{azioni}. Le azioni sono quote di partecipazione liberamente trasferibili. Possibile quotazione in Borsa.
		      \item capitale minimo di 50 000 \euro.
	      \end{itemize}

	\item \textbf{Società in accomandita per azioni (s.a.p.a.):}

\end{itemize}

\textbf{PRO:}
\begin{itemize}
	\item \textbf{Responsabilità limitata} alla quota di capitale conferita: il patrimonio privato di ogni socio è inattaccabile dai creditori (a meno che non siano commessi degli illeciti).
	\item La \textbf{gestione} può essere affidata anche ai non soci.
	\item \textbf{Tassa corporate}: per profitti alti è più conveniente rispetto ad una impresa individuale.
	\item Utili possono essere distribuiti ai soci nei momenti \textbf{fiscalmente più convenienti}.

\end{itemize}

\textbf{CONTRO:}
\begin{itemize}
	\item \textbf{Adempimenti burocratici e fiscali} sono numerosi e complessi.
	\item Obbligatorio il \textbf{conferimento di capitale} iniziale.
	\item Maggiori \textbf{obblighi di trasparenza e di governance}. Vale in particolar modo per le società per azioni: chiunque può diventare proprietario e quindi è importante che l'impresa comunichi spesso e precisamente lo stato.

\end{itemize}

\subsubsection{Società cooperative}

Le società cooperative sono imprese che pur svolgendo un'attività economica \textbf{non hanno l'obiettivo di distribuire utili significativi} in capo ai soci.

\textbf{Devono reinvestire i profitti nell'attività imprenditoriale}.

Qualora dette imprese non dovessero rispettare questi requisiti perderebbero il diritto alle \textbf{importanti agevolazioni fiscali} di cui possono beneficiare.

\subsubsection{Il sitema di governance}

Il sistema di governance di una impresa collettiva si compone di 3 enti:

\begin{itemize}
	\item \textbf{Consiglio di amministrazione (CDA):} detiene il \textbf{potere esecutivo}. Viene eletto dall'assemblea degli azionisti e ha durata limitata.
	\item \textbf{Assemblea degli azionisti:} detiene il \textbf{potere deliberativo}. Può essere:
	      \begin{itemize}
		      \item \emph{Ordinaria:} viene convocata almeno una volta all'anno. Si occupa di approvare il bilancio, della spartizione degli utili, e della nomina degli amministratori.

		      \item \emph{Straordinaria:} delibera su fusioni, scissioni, aumenti del capitale o l'emissione di obbligazioni.

	      \end{itemize}
	\item \textbf{Collegio sindacale:} detiene il \textbf{potere di controllo} su operato, amministratori e corretta stesura del bilancio.
\end{itemize}


\subsection{Riassunto}

Le forme giuridiche che un'impresa può assumere sono varie:

\begin{itemize}
	\item Imprese individuali

	\item Imprese collettive
	      \begin{itemize}
		      \item Società di persone
		            \begin{itemize}
			            \item Società semplice (s.s.)
			            \item Società in nome collettivo (s.n.c.)
			            \item Società in acoomandita semplice (s.a.s.)
		            \end{itemize}

		      \item Società di capitale
		            \begin{itemize}
			            \item Società a responsabilità limitata (s.r.l.)
			            \item Società a responsabilità limitata semplificata (s.r.l.s.)
			            \item Società per azioni (s.p.a.)
			            \item Società in accomandita per azioni (s.a.p.a.)
		            \end{itemize}

		      \item Società di cooperative

	      \end{itemize}
\end{itemize}

Come scelgo la forma giuridica? Devo considerare vari elementi:

\begin{itemize}
	\item Responsibilità patrimoniale
	\item Convenienza fiscale
	\item Obblighi di natura amministrativa e contabile
	\item Eventuale trasferibilità della partecipazione
	\item Prospettive economiche e finanziarie dell'attività aziendale
	      \begin{itemize}
		      \item Condizioni personali dei soci fondatori
		      \item Rischio
		      \item Dimensione ed il tipo di finanziamenti necessari all'impresa
	      \end{itemize}
\end{itemize}

\textbf{La forma giuridica può cambiare nel corso di vita dell'impresa!}

\end{document}