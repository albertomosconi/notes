\documentclass[../main.tex]{subfiles}

\begin{document}
\section{Le Imprese}
	
Utilizza come input beni (materie prime) e servizi, e trasforma gli input in output utilizzando delle risorse.\\
Le risorse possono essere:
    
    \begin{itemize}
        \item impianti, macchinari (capitale fisico)
        \item personale (capitale umano)
        \item conoscenze tecnologiche, brevetti (risorse immateriali)\\
            non sono risorse fisiche ma possono dare grandi vantaggi competitivi. Ad esempio con un brevetto le altre aziende non possono usare liberamente le mie tecnologie.
    \end{itemize}

L'azienda vende il proprio output \textbf{ai consumatori o altre imprese}.\\
\textbf{Finanzia le proprie attività} coi soldi degli imprenditori (ma non solo).


\subsection{Obiettivi dell'impresa}

Obiettivo principale dell'impresa (\emph{for profit}) è \textbf{generare valore (ricchezza)} per i soggetti coinvolti in essa.
$$ UTILE = RICAVI - COSTI $$

\subsubsection{Obiettivi intermedi}

    \begin{itemize}
        \item riduzione costi
        \item aumento quote di mercato
        \item miglioramento qualità prodotto
        \item internazionalizzazione o ingresso in nuovi mercati
        \item ...
    \end{itemize}

In realtà non è importante solo massimizzare il profitto ma bisogna anche prestare attenzione al contesto sociale.\\
Mostrare attenzione per il contesto sociale migliora la visibilità dell'azienda e di conseguenza aumenta il valore stesso del brand.\\
Avere comportamenti ambigui eticamente può avere ripercussioni negative (Nike che sfrutta i bambini nei paesi in evoluzione).



\end{document}