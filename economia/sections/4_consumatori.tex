\documentclass[../main.tex]{subfiles}

\begin{document}
\section{Consumatori}

I \textbf{consumatori} sono agenti economici disposti a pagare per acquistare beni o servizi.

Così come le imprese mirano a massimizzare il profitto, i \textbf{consumatori acquistano beni o servizi per aumentare il proprio benessere (utilità)}

\subsection{Utilità}

\textbf{Utilità}: misura della soddisfazione che si ricava dal consumo di beni e servizi.

La \textbf{funzione di utilità} descrive come varia il livello di soddisfazione del consumatore al variare delle quantità di beni e servizi consumati. Tipicamente valgono le seguenti condizioni:
\begin{itemize}
\item \textbf{Utilità monotona e crescente}: il consumo di un determinato bene fa aumetare l'utilità al consumatore.
\item \textbf{Utilità marginale decrescente}: l'utilità addizionale (marginale) di ogni successiva unità di consumo è via via minore. Esempio: la prima birra la bevo molto volentieri, la seconda la gradisco comunque ma leggermente meno, alla decima non la gradisco proprio.
\end{itemize}

\subsection{Prezzo di riserva}

\textbf{Prezzo di riserva (PR)}: prezzo massimo che un consumatore è disposto a pagare per acquistare un'unità di un bene.

Il prezzo di riserva guida le decisioni di acquisto:
\begin{itemize}
\item PR $>$ Prezzo praticato da imprese produttrici $\rightarrow$ Acquisto
\item PR $<$ Prezzo praticato da imprese produttrici $\rightarrow$ Non acquisto
\end{itemize}

Conoscere il prezzo di riserva consente di costruire la curva di domanda individuale.

\subsection{Curva di domanda individuale}

\textbf{Curva di domanda individuale di un bene x}: esprime, per ciascun consumatore, il prezzo di riserva di diverse quantità di x.

La curva di domanda individuale è \textbf{decrescente}. Se il prezzo sale, la quantità domandata dal consumatore scende, e viceversa.

Il prezzo di riserva è legato all'utilità marginale:
\begin{itemize}
\item  il prezzo di riserva dipende dalla quantità di bene già consumata.
\item la variazione di utilità in seguito al consumo di un'unità aggiuntiva del bene (unità marginale) è decrescente.
\end{itemize}

La curva di domanda individuale consente di valutare il \textbf{surplus del consumatore}: differenza fra il prezzo che un consumatore è disposto a pagare e il prezzo di mercato del bene. Esempio: se sono disposto a pagare 5 euro per una birra, e la vendono a 3 euro, il surplus è 2 euro.\\

\subsubsection{Determinanti della domanda individuale:}
\begin{enumerate}
\item Caratteristiche del consumatore
	\begin{itemize}
	\item \textbf{Gusti e necessità}
	\item \textbf{Reddito o ricchezza}: questo può influenzare la domanda in due modi diversi a seconda del tipo di bene:
		\begin{itemize}
		\item Beni normali: la quantità domandata aumenta all'aumentare del reddito.
		\item Beni inferiori: la quantità domandata diminuisce all'aumentare del reddito.
		\end{itemize}
	\end{itemize}
\item Caratteristiche del bene
	\begin{itemize}
	\item \textbf{Prezzo e disponibilità di beni sostituti}, ovvero beni che espletano funzioni simili a quelle di x. Se aumenta (diminuisce) il prezzo di un sostituto di x, la quantità domandata di x aumenta (diminuisce).
	\item \textbf{Prezzo e disponibilità di beni complementari}, ovvero beni che tendono ad essere consumati insieme ad x. Se aumenta (diminuisce) il prezzo di un bene complementare a x, la quantità domandata di x diminuisce (aumenta).
	\end{itemize}
\end{enumerate}

\subsection{Domanda di mercato}

\textbf{Domanda di mercato} (domanda aggregata): somma, per tutti gli N consumatori, delle quantità domandate individuali
$$Q(p) = \sum_{i=1}^N\,q_i(p)$$

La domanda di mercato può avere varie forme funzionali, ma è comunque (quasi) sempre descrescente.

\subsubsection{Elasticità della domanda}

L'elasticità della domanda è la variazione percentuale della quantità domandata al variare di una delle sue componenti: prezzo del bene, prezzo degli altri beni e reddito del consumatore.

Una misurazione accurata della variazione della domanda consente di conoscere le reazioni dei consumatori e quindi l'impatto che tali variazioni hanno sui ricavi dell'impresa.\\

\textbf{Elasticità della domanda al prezzo} del bene x: variazione percentuale della quantità domandata del bene x a seguito della variazione percentuale del suo prezzo.
$$
\varepsilon_x=\frac{\frac{\Delta q_x}{q_x}}{\frac{\Delta p_x}{p_x}}\Rightarrow\varepsilon_x=\frac{\Delta q_x}{\Delta p_x}\frac{p_x}{q_x}\Rightarrow\varepsilon_x=\frac{\partial q_x}{\partial p_x}\frac{p_x}{q_x}
$$

è in genere negativa, si considera solo il valore assoluto
$$
\varepsilon_x = |\frac{\partial q_x}{\partial p_x}\frac{p_x}{q_x}|
$$

Esempio di bene caratterizzato da bassa elasticità della domanda al prezzo: acqua (in generale tutti i beni di prima necessità).

\begin{itemize}
\item La domanda di un bene con \textbf{pochi sostituti} è \textbf{poco elastica (anelastica)}
\item La domanda di un bene con \textbf{molti sostituti} è \textbf{molto elastica (elastica)}\\
\end{itemize}

\textbf{Elasticità incrociata} del bene x rispetto a y: variazione percentuale della quantità domandata di x rispetto ad una variazione percentuale del prezzo del bene y
$$\varepsilon_{xy}=\frac{\partial q_x}{\partial p_y}\frac{p_y}{q_x}$$
Il segno dipende dalle relazioni di complementarietà e sostituibilità tra i beni
\begin{itemize}
\item \textbf{Beni complementari}: elasticità incrociata \textbf{negativa}.
\item \textbf{Beni sostituti}: elasticità incrociata \textbf{positiva}.\\
\end{itemize}

\textbf{Elasticità della domanda al reddito} del bene x: variazione percentuale della quantità domandata del bene x a seguito della variazione percentuale del reddit M
$$\varepsilon_M=\frac{\partial q_x}{\partial M} \frac{M}{q_x}$$

\begin{itemize}
\item \textbf{Beni normali}: elasticità della domanda al reddito \textbf{positiva}.
\item \textbf{Beni inferiori}: elasticità della domanda al reddito \textbf{negativa}.
\end{itemize}

\end{document}