\documentclass[../main.tex]{subfiles}

\begin{document}

La capacità delle imprese di massimizzare il profitto dipende da \textbf{fattori di mercato} quali
\begin{itemize}
    \item \textbf{Numero di concorrenti} (imprese che producono beni che i consumatori percepiscono come stretti sostituti).
    \item \textbf{Natura del prodotto} (omogeneo vs. differenziato).
    \item \textbf{Grado di libertà di entrata} (o uscita) delle imprese nel mercato.
    \item \textbf{Quantità dell'informazione} detenuta da imprese e consumatori.
    \item \dots
\end{itemize}

Tali caratteristiche definiscono la \textbf{forma di mercato (livello di competizione)} in cui opera l'impresa.

I mercati si collocano in un continuum tra concorrenza perfetta e monopolio
\begin{itemize}
    \item \textbf{Concorrenza perfetta}: infinite imprese nell'azienda, massimo livello di competizione.
    \item \textbf{Monopolio}: una sola impresa nell'industria, minimo livello di competizione.
\end{itemize}

\subsubsection{Concorrenza perfetta}

Il modello di concorrenza perfetta si basa su quattro ipotesi fondamentali.
\begin{enumerate}
    \item Esiste un \textbf{numero molto elevato di imprese} nel mercato; la singola impresa produce una quota trascurabile dell'offerta totale.
    \item Tutte le imprese producono un \textbf{prodotto identico}; in altre parole, il prodotto è omogeneo (non differenziato).
    \item Acquirenti e venditori hanno una \textbf{conoscenza perfetta} dei prodotti e dei prezzi.
    \item Esiste \textbf{completa libertà di entrata e di uscita} da parte di nuove imprese.
\end{enumerate}

Non esiste un mercato che soddisfa perfettamente le condizioni elencate sopra, ma ci sono esempi che si avvicinano: \emph{mercato ortofrutticolo}, molti produttori, i prodotti venduti sono tutti gli stessi. Queste attività non sono particolarmente lucrative, fare profitti elevati è difficile.

La concorrenza perfetta è una \textbf{forma di mercato estrema}:
\begin{itemize}
    \item Le imprese non hanno alcun potere di influenzare il prezzo del prodotto.
    \item Il prezzo a cui vendono è determinato dall'interazione della domanda e dell'offerta complessiva di mercato (si veda dopo).
\end{itemize}

In altri termini le \textbf{imprese sono price-taker}
\begin{itemize}
    \item Se fissassero un \textbf{prezzo superiore} a quello di mercato, \textbf{non venderebbero nulla}.
    \item Se fissassero un \textbf{prezzo inferiore} a quello di mercato, \textbf{non avrebbero la capacità di soddisfare l'intero mercato}. Gli altri produttori continuano a vendere al loro prezzo e fanno più soldi.
\end{itemize}

\textbf{Qual è la quantità q che consente all'impresa di massimizzare il profitto (= ricavi - costi)?} \\

\textbf{Curva di offerta individuale}: esprime, per ogni livello di prezzo p, la quantità ottimale q di produzione del bene.

Essendo l'impresa price-taker, p non dipende dalla quantità prodotta dalla singola impresa q:
$$RT(q) = p\cdot q$$
$$RM(q) = p$$

\textbf{Condizione di massimizzazione del profitto}: $RM(q) = CM(q)$. Si ottiene quindi:
$$p = CM(q)$$

\textbf{Condizione minima di produzione}: $\pi = pq - CT(q) > 0$.

Si ottiene quindi che il prezzo deve essere superiore al costo medio affinchè l'impresa sia in grado di ottenere profitti positivi.
$$p > \frac{CT(q)}{q}$$

\textbf{Curva di offerta di mercato}: somma delle curve di offerta di tutte le imprese nel mercato.

In un mercato in cui operano molteplici aziende produttrici, il \textbf{prezzo di equilibrio} dipende dall'\textbf{incontro tra domanda e offerta di mercato}.

\begin{itemize}
    \item \textbf{Se p $>$ prezzo di equilibrio di mercato} si verifica un \textbf{eccesso di offerta}, ovvero alcuni produttori non riescono a vendere. Di conseguenza il prezzo si abbassa per vendere di più, anche ai consumatori con prezzo di riserva più basso.
    \item \textbf{Se p $<$ prezzo di equilibrio di mercato} si verifica un \textbf{eccesso di domanda}, ovvero alcuni consumatori sarebbero disposti a comprare il bene ma questo non sarebbe disponibile.
\end{itemize}

Nel lungo periodo, se le imprese già operative ottengono profitti positivi (p $>$ costo medio), \textbf{nuove imprese saranno attirate nel mercato}.

\begin{itemize}
    \item nuove imprese entrano nel mercato attratte dal profitto.
    \item l'offerta sale e il prezzo di equilibrio scende.
    \item per alcune imprese diviene p $<$ costo medio.
    \item le imprese con costo medio $>$ p escono dal mercato.
    \item l'offerta scende e il prezzo sale
    \item \dots e così via.
\end{itemize}

\textbf{Equilibrio di lungo periodo}: entrata e uscita cessano quando non sono più possibili profitti. Rimangono sul mercato solo le imprese più efficienti che producono al costo medio minimo. Le \textbf{imprese conseguono profitti nulli}.

\textbf{Dal punto di vista dell'impresa, la concorrenza non è desiderabile.}

\subsubsection{Monopolio}

A volte in un mercato c'è un'\textbf{unica impresa produttrice} (monopolista). Perchè?
\begin{itemize}
    \item Esistono ostacoli insormontabili (le barriere all'entrata) che impediscono ad altre imprese di entrare e competere.
\end{itemize}

Il monopolista è \textbf{price-maker}:
\begin{itemize}
    \item A differenza della concorrenza perfetta, \textbf{fronteggia l'intera curva di domanda di mercato}.
    \item \textbf{Il prezzo al quale egli vende il prodotto non è indipendente dalla quantità venduta.}
\end{itemize}

Ne consegue che i ricavi totali sono dati da:
$$RT(q) = p(q)\cdot q$$
Ed il ricavo marginale è quindi:
$$
RM(q)=\frac{\partial p(q)}{\partial q}\cdot q + p(q) = p(q)\cdot \left(\frac{\partial p(q)}{\partial q}\frac{q}{p(q)}+1\right)=p(q)\cdot \left(-\frac{1}{\varepsilon}+1\right)
$$
Dove $\varepsilon$ è l'elasticità della domanda al prezzo (in valore assoluto).

Si applica la \textbf{condizione di massimizzazione del profitto}: $RM(q) = CM(q)$, e si ottiene quindi
$$p(q) \left(-\frac{1}{\varepsilon}+1\right)=CM(q)$$
$$\frac{p(q)-CM(q)}{p(q)} = \frac{1}{\varepsilon}$$

Il monopolista fissa un prezzo al di sopra dei costi marginali (si dice che ha \textbf{potere di mercato}). Il potere di mercato è tanto maggiore quanto meno la domanda risponde alle variazioni di prezzo.

\end{document}