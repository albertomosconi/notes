\documentclass[12pt, letterpaper]{article}

\usepackage[margin=15mm, includefoot]{geometry}

\setlength{\parskip}{0mm}
\setlength{\parindent}{0em}

\begin{document}

\section{Definizioni Giuridiche}

	\subsection{Attività}
	
		\begin{itemize}
			\item \textbf{Economica:}
			
				l'output deve poter essere oggetto di \textbf{scambio} su un mercato (deve avere un valore \textbf{economico}).
			
			\item \textbf{Professionale:}
			
				svolta abitualmente, ma non necessariamente:
				\begin{itemize}
					\item con continuità temporale
					\item in esclusiva
					\item dall'imprenditore (è possibile delegare)
				\end{itemize}
			
			\item \textbf{Organizzata:}
			
				l'impresa ha una sua organizzazione, struttura che consente una \textbf{gestione coordinata delle risorse}.\\		
				l'imprenditore organizza liberamente l'impresa.
			
		\end{itemize}	
	
	\subsection{Impresa vs Società vs Azienda vs Ditta}
		
		Le imprese possono essere società, ma non necessariamente\\		
		Azienda e ditta hanno altri significati rispetto ad impresa
		
		\begin{itemize}
			\item \textbf{Società:}\\			
				contratto con cui due o più persone conferiscono beni o servizi per l'esercizio in comune di una attività economica allo scopo di dividerne gli utili
			
			\item \textbf{Azienda:}\\			
				complesso dei beni organizzati dall'imprenditore per l'esercizio dell'impresa.
				
			\item \textbf{Ditta:}\\			
				nome commerciale scelto dall'imprenditore per esercitare l'impresa.\\				
				ha valore commerciale (google, apple, ferrari), per questo, la legge ne garantisce \textbf{l'uso esclusivo}.
				
		\end{itemize}
		
	\section{Impresa}
	
		Utilizza com input beni (materie prime) e servizi, e trasforma gli input in output utilizzando delle risorse.\\
		Le risorse possono essere:
			
			\begin{itemize}
				\item impianti, macchinari (capitale fisico)
				\item personale (capitale umano)
				\item conoscenze tecnologiche, brevetti (risorse immateriali)\\
					non sono risorse fisiche ma possono dare grandi vantaggi competitivi. Ad esempio con un brevetto le altre aziende non possono usare liberamente le mie tecnologie.
			\end{itemize}
	
		L'azienda vende il proprio output \textbf{ai consumatori o altre imprese}.\\
		\textbf{Finanzia le proprie attività} coi soldi degli imprenditori (ma non solo).
		
		
	\subsection{Obiettivi dell'impresa}
	
		Obiettivo principale dell'impresa (\emph{for profit}) è \textbf{generale valore (ricchezza)} per i soggetti coinvolti in essa.
		$$ UTILE = RICAVI - COSTI $$
		
		\subsubsection{Obiettivi intermedi}
		
			\begin{itemize}
				\item riduzione costi
				\item aumento quote di mercato
				\item miglioramento qualità prodotto
				\item internazionalizzazione o ingresso in nuovi mercati
				\item ...
			\end{itemize}
		
		In realtà non è importante solo massimizzare il profitto ma bisogna anche prestare attenzione al contesto sociale.\\
		Mostrare attenzione per il contesto sociale migliora la visibilità dell'azienda e di conseguenza aumenta il valore stesso del brand.\\
		Avere comportamenti ambigui eticamente può avere ripercussioni negative (Nike che sfrutta i bambini nei paesi in evoluzione).
		
		
		
		
		
		
		
		
		
		
		
		
		
	
\end{document}