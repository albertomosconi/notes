\documentclass[../main.tex]{subfiles}

\begin{document}

\subsection{I rischi delle Imprese}
		
\textbf{Rischio:} impossibilità di prevedere con certezza gli esiti futuri delle decisioni in merito alle attività dell'impresa.
	
Non esiste impresa senza rischio.

\subsubsection{Fattori di Rischio}

\begin{itemize}
\item
\textbf{Tempo:} i risultati si vedono domani. Mancano alcune informazioni necessarie per decidere.

\item
\textbf{Rigidità strutturale:} l'impresa ha un'organizzazione non immediatamente modificabile in risposta all'ambiente.
	\begin{itemize}
	\item
	Esempio: in caso di riduzione della domanda non sempre è possibile ridurre il costo del personale.
	\end{itemize}

\item
\textbf{Contesto dinamico e mutevole:} domanda, preferenze dei consumatori, numero e tipologia di concorrenti, tecnologie, condizioni di accesso al credito... \textbf{sono variabili nel tempo}.

	\begin{itemize}
	\item
	Esempio: Nokia fino al 2006 era il maggiore produttore di telefoni. Nel 2007 Apple entra nel mercato con l'iPhone. Nokia è stata molto lenta a rispondere e quindi ha perso una grandissima fetta di mercato.
	\end{itemize}

\end{itemize}

\subsubsection{L'imprenditore si assume il rischio di impresa}
Cosa significa?

\begin{itemize}
\item
\textbf{Accezione positiva:} si appropria dei guadagni (profitti positivi)
\item
\textbf{Accezione negativa:} risponde delle perdite (profitti negativi)
\end{itemize}

Come risponde ad un evento negativo? Dipende dell'assetto proprietario / forma giuridica:
\begin{itemize}
\item
\textbf{Responsibilità illimitata (personale):} l'limprenditore  (i soci) risponde (rispondono) con tutto il proprio patrimonio personale.
\item
\textbf{Responsibilità limitata:} l'imprenditore (i soci) risponde (rispondono) con i soli capitali conferiti.
\end{itemize}

\subsection{Come nasce una Impresa}

Per fondare una impresa è necessario capitale proprio?
\begin{itemize}
\item
\textbf{In linea di principio no:} l'imprenditore potrebbe raccogliere capitale da soci esterni (\textbf{capitale di rischio}) e/o credito (\textbf{capitale di debito}) sulla base della sua idea di business.

\item
Tuttavia \textbf{la presenza di capitale proprio} dei fondatori garantisce i creditori da rischio di insolvenza e \textbf{segnala credibilmente il valore dell'idea di business} a finanziatori esterni.
\end{itemize}

\textbf{Business plan:} descrizione dell'idea imprenditoriale in cui si dimostra che l'attività proposta merita fiducia più di altre possibilità di investimento. Contiene informazioni su:
\begin{itemize}
\item Il \textbf{prodotto o il servizio} che si intende offrire.
\item Il \textbf{mercato} in cui l'impresa andrà ad operare.
\item La \textbf{strategia} e l'implementazione della stessa.
\item Il \textbf{gruppo dirigente}, ad esempio se ci sono persone che hanno già molta esperienza sul mercato.
\item Le \textbf{previsioni finanziarie}.
\end{itemize}

\subsection{Come muore una Impresa}

\textbf{L'impresa ha durata indefinita}, non muore con l'imprenditore.

\begin{itemize}
\item General Electric: fondata del 1882, posizione di rilievo dal 1917
\item Apple: esiste ancora dopo la morte di Steve Jobs
\end{itemize}

\textbf{Rischia di ``morire'' se non realizza profitti} e dunque non riesce a remunerare i fattori produttivi
\begin{itemize}
\item In genere la vita media di una impresa è inferiore a quella di una persona.
\item In Italia le imprese vivono in media 12 anni.
\end{itemize}

\subsubsection{Le cause della morte}

L'azienda può ''morire''  in vari modi:

\begin{itemize}
\item \textbf{Fallimento:} scioglimento coatto - l'impresa è sciolta per ordine del tribunale, i suoi beni vengono venduti per ripagare i debiti (asta giudiziaria)
\item \textbf{Liquidazione:} scioglimento volontario - vendita volontaria dei beni decisa dai soci. NB: la morte per liquidazione non sempre ha un'accezione negativa.
\item \textbf{Acquisizione/Fusione:} l'impresa viene assorbita da un'altra impresa. NB: la morte per fusione ha spesso un'accezione positiva.
\end{itemize}

\subsection{Le Tipologie di Imprese}


\end{document}
























