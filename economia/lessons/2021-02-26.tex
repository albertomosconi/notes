\documentclass[../main.tex]{subfiles}

\begin{document}

\subsection{I rischi delle imprese}
		
\textbf{Rischio:} impossibilità di prevedere con certezza gli esiti futuri delle decisioni in merito alle attività dell'impresa.
	
Non esiste impresa senza rischio.

\subsubsection{Fattori di rischio}

\begin{itemize}
\item
\textbf{Tempo:} i risultati si vedono domani. Mancano alcune informazioni necessarie per decidere.

\item
\textbf{Rigidità strutturale:} l'impresa ha un'organizzazione non immediatamente modificabile in risposta all'ambiente.
	\begin{itemize}
	\item
	Esempio: in caso di riduzione della domanda non sempre è possibile ridurre il costo del personale.
	\end{itemize}

\item
\textbf{Contesto dinamico e mutevole:} domanda, preferenze dei consumatori, numero e tipologia di concorrenti, tecnologie, condizioni di accesso al credito... \textbf{sono variabili nel tempo}.

	\begin{itemize}
	\item
	Esempio: Nokia fino al 2006 era il maggiore produttore di telefoni. Nel 2007 Apple entra nel mercato con l'iPhone. Nokia è stata molto lenta a rispondere e quindi ha perso una grandissima fetta di mercato.
	\end{itemize}

\end{itemize}

\subsubsection{L'imprenditore si assume il rischio di impresa}
Cosa significa?

\begin{itemize}
\item
\textbf{Accezione positiva:} si appropria dei guadagni (profitti positivi)
\item
\textbf{Accezione negativa:} risponde delle perdite (profitti negativi)
\end{itemize}

Come risponde ad un evento negativo? Dipende dell'assetto proprietario / forma giuridica:
\begin{itemize}
\item
\textbf{Responsibilità illimitata (personale):} l'limprenditore  (i soci) risponde (rispondono) con tutto il proprio patrimonio personale.
\item
\textbf{Responsibilità limitata:} l'imprenditore (i soci) risponde (rispondono) con i soli capitali conferiti.
\end{itemize}

\subsection{Come nasce una impresa}

Per fondare una impresa è necessario capitale proprio?
\begin{itemize}
\item
\textbf{In linea di principio no:} l'imprenditore potrebbe raccogliere capitale da soci esterni (\textbf{capitale di rischio}) e/o credito (\textbf{capitale di debito}) sulla base della sua idea di business.

\item
Tuttavia \textbf{la presenza di capitale proprio} dei fondatori garantisce i creditori da rischio di insolvenza e \textbf{segnala credibilmente il valore dell'idea di business} a finanziatori esterni.
\end{itemize}

\textbf{Business plan:} descrizione dell'idea imprenditoriale in cui si dimostra che l'attività proposta merita fiducia più di altre possibilità di investimento. Contiene informazioni su:
\begin{itemize}
\item Il \textbf{prodotto o il servizio} che si intende offrire.
\item Il \textbf{mercato} in cui l'impresa andrà ad operare.
\item La \textbf{strategia} e l'implementazione della stessa.
\item Il \textbf{gruppo dirigente}, ad esempio se ci sono persone che hanno già molta esperienza sul mercato.
\item Le \textbf{previsioni finanziarie}.
\end{itemize}

\subsection{Come muore una impresa}

\textbf{L'impresa ha durata indefinita}, non muore con l'imprenditore.

\begin{itemize}
\item General Electric: fondata del 1882, posizione di rilievo dal 1917
\item Apple: esiste ancora dopo la morte di Steve Jobs
\end{itemize}

\textbf{Rischia di ``morire'' se non realizza profitti} e dunque non riesce a remunerare i fattori produttivi
\begin{itemize}
\item In genere la vita media di una impresa è inferiore a quella di una persona.
\item In Italia le imprese vivono in media 12 anni.
\end{itemize}

\subsubsection{Le cause della morte}

L'azienda può ''morire''  in vari modi:

\begin{itemize}
\item \textbf{Fallimento:} scioglimento coatto - l'impresa è sciolta per ordine del tribunale, i suoi beni vengono venduti per ripagare i debiti (asta giudiziaria)
\item \textbf{Liquidazione:} scioglimento volontario - vendita volontaria dei beni decisa dai soci. NB: la morte per liquidazione non sempre ha un'accezione negativa.
\item \textbf{Acquisizione/Fusione:} l'impresa viene assorbita da un'altra impresa. NB: la morte per fusione ha spesso un'accezione positiva.
\end{itemize}

\subsection{Le tipologie di imprese}

\subsubsection{Dimensioni di classificazione}

Le imprese si classificano secondo varie dimensioni:

\begin{enumerate}
\item \textbf{Proprietà:} pubblica (il proprietario è un ente pubblico, es: lo Stato) o privata.

\item \textbf{Obiettivo:} profit (obiettivo principale è il profitto) o no profit (l'obiettivo è uno scopo alternativo, spesso socialmente rilevante)

\item \textbf{Dimensione:} addetti e fatturato
	\begin{itemize}
	\item
	\textbf{Grandi:} addetti >= 250, fatturato > 50 mil. euro
	\item \textbf{Medie:} addetti 50-249, fatturato 10-50 mil. euro
	\item \textbf{Piccole:} addetti < 50, fatturato < 10 mil. €
	\item \textbf{Microimprese:} addetti < 10, fatturato <= 2 mil. euro. In Italia il 95\% delle aziende attiva è una microimpresa.
	\end{itemize}

\item \textbf{Tipologia di output:}
	\begin{itemize}
	\item
	\textbf{Beni Materiali:} imprese agricole (producono beni) o industriali/manufatturiere (compiono trasformazioni tecniche dei beni)
	
	\item \textbf{Servizi:} trasporto e telecomunicazioni, distribuzione energia elettrica, etc. Queste sono la maggiorparte.
	\end{itemize}

\item \textbf{Numero di output}:
	\begin{itemize}
	\item \textbf{Monoprodotto:} imprese che producono/vendono un solo prodotto.
	\item  \textbf{Diversificate:} imprese che producono/vendono vari prodotti/servizi da qualche punto di vista imparentati tra loro.
	\item \textbf{Conglomerati:} imprese che producono/vendono vari prodotti/servizi poco imparentati tra loro. Spesso esiste un core business (prodotto/servizio ritenuto più importante)
		\begin{itemize}
		\item Esempio: Alphabet, parent company di Google
		\end{itemize}
	\end{itemize}

\item \textbf{Consumatore:}
	\begin{itemize}
	\item
	\textbf{Wholesale (all'ingrosso):} imprese che producono e vendono prodotti intermedi ad altre imprese che, a loro volta, li utilizzano nel loro processo produttivo.
	\item \textbf{Retail (al dettaglio):} imprese che producono e vendono il prodotto consumatore in un mercato finale.
	\end{itemize}

\item \textbf{Localizzazione delle attività produttive:} sono \textbf{multinazionali}, che hanno interessi economici e attività produttive in più nazioni, o \textbf{nazionali}.

\end{enumerate}

\subsection{Settore}

Un settore è l'insieme di tutte le imprese che producono beni (erogano servizi) che i consumatori considerano \textbf{sostituti}, ovvero che soddisfano un bisogno simile.

Un dato settore può essere disaggregato: Manifattura -> Aerreonautica -> Velivoli per uso civile.

Esiste una classificazione settoriale standard, in Italia è ATECO. Ogni impresa deve dichiarare precisamente il proprio settore.


\section{Forme Giuridiche}

La forma giuridica definisce quali sono i diritti e doveri di una impresa.

Esempi di \textbf{diritti di una impresa}:
\begin{itemize}
\item
Diritti di proprietà su beni e servizi utilizzati per l'esercizio dell'attività.
\item Diritto di affittare un immobile (essere intestataria del contratto di affitto)
\item Diritto a difendersi in tribunale in caso di controversie.
\end{itemize}

Esempi di \textbf{doveri di una impresa}:
\begin{itemize}
\item Dovere di redigere il bilancio.
\item Dovere di rispettare le leggi a tutela dei lavoratori
\item Dovere di rispettare la normativa ambientale.
\end{itemize}

Per \textbf{forma giuridica} si intende la \textbf{tipologia giuridica del soggetto a cui fa capo l'attività} e le norme ad essa conseguenti.

La scelta della forma giuridica è importante perchè determina una serie di diritti ... ?????????????

Il codice civile distingue

\begin{enumerate}
\item \textbf{Imprese individuali:} costituite  da un'unica persona fisica. Non c'è distinzione giuridica tra il proprietario e l'impresa stessa.

\item \textbf{Imprese collettive:} società
\end{enumerate}

\subsection{Imprese individuali}

\textbf{Il titolare è illimitatamente responsabile delle obbligazioni} dell'impresa con tutto il patrimonio personale.

Tipica di attività quali: commercialista, architetto, ingegnere, medico, consulente di vario genere.

\textbf{Impresa familiare:} estensione dell'impresa individuale, quando l'imprenditore si avvale in modo continuativo della prestazione lavorativa dei familiari. \\

\textbf{PRO:}
\begin{itemize}
\item \textbf{Semplicità} nella costituzione e lo scioglimento dell'impresa. Non è richiesto il versamento del capitale.

\item \textbf{Pochi obblighi contabili}, non sono obbligate a redigere un bilancio in forma complessa.

\item \textbf{Autonomia e velocità} decisionale.
\end{itemize}

\textbf{CONTRO:}
\begin{itemize}
\item \textbf{Responsibilità illimitata:} l'imprenditore risponde con il proprio patrimonio personale per le obbligazioni assunte nel corso dell'attività.

\item In caso di forti guadagni le \textbf{imposte} crescono (si applicano aliquote progressive previste dall'Irpef).

\end{itemize}

\subsection{Imprese collettive}

Le imprese collettive si distinguono principalmente in:

\begin{itemize}
\item \textbf{Società di persone}
Soci hanno responsabilità solidale e illimitata per le obbligazioni sociali (con alcune eccezioni). In queste società i soci sono tassati come se fossero individui.

\item \textbf{Società di capitale}
Soci hanno responsabilità limitata e circoscritta ai loro rispettivi conferimenti sociali.
C'è distinzione giuridica tra proprietario e impresa.

\item \textbf{Società di cooperative} Soci hanno responsabilità limitata e circoscritta ai loro rispettivi conferimenti sociali.
Si contraddistinguono per lo scopo mutualistico.
\end{itemize}

\subsubsection{Società di persone}

\begin{itemize}
\item
Semplice, riservata ad attività economiche non commerciali (attività agricole e per la gestione di patrimoni immobiliari)
\item Società in nome collettivo (s.n.c.): può esercitare sia attività di impresa commerciale, sia attività economiche non commerciali.
\item Società in accomandita semplice (s.a.s.)

	\begin{itemize}
	\item \textbf{soci accomandatari:} si assumono in forma illimitata e solidale le responsabilità connesse all'esercizio dell'impresa.
	\item \textbf{soci accomandanti:} affidano in gestione i loro capitali ad altri soci e sono responsabili sono del capitale conferito. Per questi soci vale la responsabilità limitata.
	
	\end{itemize}
\end{itemize}

\textbf{PRO:}
\begin{itemize}
\item Costituzione e la tenuta della contabilità relativamente semplici.

\item \textbf{Procedure} burocratiche, fiscali, contabili e tributarie minime.

\item \textbf{Non è obbligatorio il versamento di un capitale minimo} da parte dei soci (l'importo è stabilito dal contratto sociale).

\item Più facile espandersi e trovare capitale addizionale.

\end{itemize}

\textbf{CONTRO:}
\begin{itemize}
\item \textbf{Responsabilità illimitata} (a parte accomandanti della s.a.s.) e \textbf{solidale}: in caso di fallimento, i creditori possono rivalersi sul patrimonio privato di ciascun membro della società. \textbf{Se un socio non adempie, il debito dovrà essere saldato dagli altri!}
\item Minore autonomia decisionale, problemi di \textbf{coordinamento}.

\end{itemize}

\subsubsection{Società di capitali}

\begin{itemize}
\item \textbf{Società a responsabilità limitata (s.r.l.):} il capitale sociale (la proprietà) è diviso in \textbf{quote} (le quote non possono essere oggetto di sollecitazione all'investimento del pubblico risparmio). C'è un capitale minimo di 10 000 €.

\item \textbf{Società a responsabilità limitata semplificata (s.r.l.s.):} forma di s.r.l. recentemente introdotta per favorire ??????. Capitale minimo di 1 €.

\item \textbf{Società per azioni (s.p.a.):}

\begin{itemize}
\item
Il pratimonio sociale è costituito da \textbf{azioni}. Le azioni sono quote di partecipazione liberamente trasferibili. Possibile quotazione in Borsa.
\item capitale minimo di 50 000 €.
\end{itemize}

\item \textbf{Società in accomandita per azioni (s.a.p.a.):}

\end{itemize}

\textbf{PRO:}
\begin{itemize}
\item \textbf{Responsabilità limitata} alla quota di capitale conferita: il patrimonio privato di ogni socio è inattaccabile dai creditori (a meno che non siano commessi degli illeciti).
\item La \textbf{gestione} può essere affidata anche ai non soci.
\item \textbf{Tassa corporate}: per profitti alti è più conveniente rispetto ad una impresa individuale.
\item Utili possono essere distribuiti ai soci nei momenti \textbf{fiscalmente più convenienti}.

\end{itemize}

\textbf{CONTRO:}
\begin{itemize}
\item \textbf{Adempimenti burocratici e fiscali} sono numerosi e complessi.
\item Obbligatorio il \textbf{conferimento di capitale} iniziale.
\item Maggiori \textbf{obblighi di trasparenza e di governance}. Vale in particolar modo per le società per azioni: chiunque può diventare proprietario e quindi è importante che l'impresa comunichi spesso e precisamente lo stato.

\end{itemize}

\subsubsection{Società cooperative}

Le società cooperative sono imprese che pur svolgendo un'attività economica \textbf{non hanno l'obiettivo di distribuire utili significativi} in capo ai soci.

\textbf{Devono reinvestire i profitti nell'attività imprenditoriale}.

Qualora dette imprese non dovessero rispettare questi requisiti perderebbero il diritto alle \textbf{importanti agevolazioni fiscali} di cui possono beneficiare.

\subsection{Riassunto}

Come scelgo la forma giuridica? Devo considerare vari elementi:

\begin{itemize}
\item Responsibilità patrimoniale
\item Convenienza fiscale
\item Obblighi di natura amministrativa e contabile
\item Eventuale trasferibilità della partecipazione
\item Prospettive economiche e finanziarie dell'attività aziendale
\begin{itemize}
\item Condizioni personali dei soci fondatori
\item Rischio
\item Dimensione ed il tipo di finanziamenti necessari all'impresa
\end{itemize}
\end{itemize}

\textbf{La forma giuridica può cambiare nel corso di vita dell'impresa!}

\end{document}
























