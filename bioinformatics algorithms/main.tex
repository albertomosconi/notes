\documentclass[12pt, letterpaper]{article}
\usepackage[utf8]{inputenc}
\usepackage[margin=15mm, includefoot]{geometry}

\usepackage{eurosym}

\usepackage{titlesec}
\titleformat{\section}
{\newpage\Large\bfseries}{\thesection}{1em}{}
[\titlerule]

\renewcommand*\contentsname{Contents}

\setlength{\parskip}{1mm}
\setlength{\parindent}{0em}

\begin{document}

\begin{titlepage}
    
    \vspace*{2cm}
    
    \begin{center}
        \huge{\bfseries Bioinformatics Algorithms} \\[10pt]
        \Large{Lecture Notes} \\[10pt]
        \large{a.a. 2020/2021} \\
    \end{center}
    
    \vfill
    
    \begin{flushleft}        
        \large
        \textbf{Alberto Mosconi} \\
        \normalsize
        Politecnico di Milano
    \end{flushleft}
    
\end{titlepage}

\thispagestyle{empty}
\tableofcontents
\newpage

\setcounter{page}{1}

\section{Definitions}

\subsection{Alphabet}
Let $\Sigma$ be a finite set of symbols (alternatively called characters), called the alphabet. No assumption is made about the nature of the symbols.

\subsection{String (or word)}
A string over $\Sigma$ is any finite sequence of symbols from $\Sigma$. For example if $\Sigma = \{0,1\}$ then $01011$ is a string over $\Sigma$.

The \textbf{length of a string \emph{S} is the number of symbols} in \emph{s} (the length of the sequence) and can be any non-negative integer. It is often denoted as $|S|$.

The \textbf{empty string} is the unique string over $\Sigma$ of length 0, and is denoted as $\epsilon$.

The \textbf{set of all strings over $\mathbf{\Sigma}$ of length \emph{n}} is denoted $\mathbf{\Sigma^n}$. For example , if $\Sigma = {0, 1}$, then $\Sigma^2 = \{00, 01, 10, 11\}$. Note that $\Sigma^0 = \{\epsilon\}$ for any alphabet $\Sigma$.

The \textbf{set of all strings of any length over $\mathbf{\Sigma}$} is the \emph{Kleene closure} of $\Sigma$ and is denoted as $\mathbf{\Sigma^*}$. Also: 
$$
\Sigma^* = \bigcup_{n \in {N} \cup \{0\}} \Sigma^n
$$

The \textbf{set of all non-empty strings over $\mathbf{\Sigma}$} is denoted by $\mathbf{\Sigma^+}$

\subsection{Substrings, prefixes and suffixes}

A string \emph{s} is said to be a \textbf{substring} (or \emph{factor}) of \emph{t} if there exist (possibly empty) strings \emph{u} and \emph{v} such that \emph{t = usv}.

Given a string \emph{t}, \textbf{suffixes} and \textbf{prefixes} are special substrings of \emph{t}.

A string \emph{s} is said to be a \textbf{prefix} of \emph{t} if there exists a string \emph{u} such that \emph{t = su}. If \emph{u} is nonempty, \emph{s} is said to be a \emph{proper} prefix of \emph{t}.

Simmetrically, a string \emph{s} is said to be a \textbf{suffix} of \emph{t} if there exists a string \emph{u} such that \emph{t = us}. If \emph{u} is nonempty, \emph{s} is said to be a \emph{proper} suffix of \emph{t}.

\subsection{Reverse, palindrome and rotations}

The \textbf{reverse} of a string is a string with the same symbols but in reverse order. For example, if \emph{s = abc} (where \emph{a}, \emph{b}, and \emph{c} are symbols of the alphabet), then the reverse of \emph{s} is \emph{cba}.

A string that is the reverse of itself is called a \textbf{palindrome}, which also includes the empty string and all strings of length 1.

A string \emph{s = uv} is said to be a \textbf{rotation} of \emph{t} if \emph{t = vu}. For example, if $\Sigma = \{0, 1\}$ the string \emph{0011001} is a rotation of \emph{0100110}, where \emph{u = 00110} and \emph{v = 01}.

\section{Comparing strings}

\subsection{Hamming distance}

The \textbf{Hamming distance} between \textbf{two strings of equal length} is the \textbf{number of positions at which the corresponding symbols are different}.

\begin{itemize}
\item ``ka\emph{rol}in'' and ``ka\emph{thr}in'' $\rightarrow$ 3
\item 10\emph{1}1\emph{1}01 and 10\emph{0}1\emph{0}01 $\rightarrow$ 2
\end{itemize}

With Hamming distance we can formalize \emph{substitutions} in biological sequences - or simply sequencing errors in which the wrong base pair is identified.

\subsection{Edit distance}

The \textbf{edit distance} is a way of quantifying how dissimilar two strings (e.g., words) are to one another by counting the \textbf{minimum number of operations required to transform on string into the other}.

In the \emph{Levenshtein distance} (the most common), edit operations are: \textbf{removal}, \textbf{insertion}, and \textbf{substitution}.

The edit distance between ``kitten'' and ``sitting'' is 3. A minimal edit script that transforms the former into the latter is:

\begin{itemize}
\item \textbf{k}itten $\rightarrow$ \textbf{s}itten (substitute \emph{s} for \emph{k})
\item sitt\textbf{e}n $\rightarrow$ sitt\textbf{i}n (substitute \emph{i} for \emph{e})
\item sittin $\rightarrow$ sittin\textbf{g} (insert \emph{g} at the end)
\end{itemize}

The number of solutions (sequences of operations) is infinite.

\end{document}


















