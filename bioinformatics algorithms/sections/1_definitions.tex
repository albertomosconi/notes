\documentclass[../main.tex]{subfiles}

\begin{document}

\section{Definitions}

\subsection{Alphabet}
Let $\Sigma$ be a finite set of symbols (alternatively called characters), called the alphabet. No assumption is made about the nature of the symbols.

\subsection{Strings}
A string (or word) over $\Sigma$ is any finite sequence of symbols from $\Sigma$. For example if $\Sigma = \{0,1\}$ then $01011$ is a string over $\Sigma$.

The \textbf{length of a string \emph{S} is the number of symbols} in \emph{s} (the length of the sequence) and can be any non-negative integer. It is often denoted as $|S|$.

The \textbf{empty string} is the unique string over $\Sigma$ of length 0, and is denoted as $\varepsilon$.

The \textbf{set of all strings over $\mathbf{\Sigma}$ of length \emph{n}} is denoted $\mathbf{\Sigma^n}$. For example , if $\Sigma = {0, 1}$, then $\Sigma^2 = \{00, 01, 10, 11\}$. Note that $\Sigma^0 = \{\varepsilon\}$ for any alphabet $\Sigma$.

The \textbf{set of all strings of any length over $\mathbf{\Sigma}$} is the \emph{Kleene closure} of $\Sigma$ and is denoted as $\mathbf{\Sigma^*}$. Also:
$$
    \Sigma^* = \bigcup_{n \in {N} \cup \{0\}} \Sigma^n
$$

The \textbf{set of all non-empty strings over $\mathbf{\Sigma}$} is denoted by $\mathbf{\Sigma^+}$

\subsection{Subsequences}

A \textbf{subsequence} of S is a string that can be \textbf{obtained by deleting any number (from 0 to n) of non-consecutive characters} from S, including S and the empty string too.

\subsubsection{Let us count: subsequences}

A subsequence of S can be described by a binary string B of n elements, telling us for each position i if the corresponding character is kept in the subsequence (1) or deleted (0).

\begin{itemize}
    \item S = apple, B = 11010 $\rightarrow$ subsequence = apl
\end{itemize}

Thus, the \textbf{number of subsequences} corresponds to the number of binary strings of length n, hence
$$
    subsequences = 2^n
$$

Notice that if two subsequences produced by different choices are \textbf{identical, they will be counted twice}.

\subsection{Substrings}

A string \emph{s} is said to be a \textbf{substring} (or \emph{factor}) of \emph{t} if there exist (possibly empty) strings \emph{u} and \emph{v} such that \emph{t = usv}.

Given a string \emph{t}, \textbf{suffixes} and \textbf{prefixes} are special substrings of \emph{t}.

\subsubsection{Let us count: substrings}

Given a string of length \emph{n}:

\begin{itemize}
    \item We have 1 substring of length \emph{n} (starting at 1)
    \item We have 2 substrings of length \emph{n-1} (starting at 1 and 2)
    \item We have 3 substrings of length \emph{n-2} (starting at 1,2 and 3)
    \item ...
    \item We have $n-i+1$ substrings of length $i$ (starting at all positions from 1 to $n-i+1$)
    \item We have \emph{n} substrings of length 1 (starting at every position)
\end{itemize}

Plus, 1 substring of length 0.

This brings the \textbf{total number of substrings} of a string of length \emph{n} to
$$
    substrings = \sum^n_{i=1}i=\frac{n(n+1)}{2}+1
$$

\subsection{Prefixes and suffixes}

A string \emph{s} is said to be a \textbf{prefix} of \emph{t} if there exists a string \emph{u} such that \emph{t = su}. If \emph{u} is nonempty, \emph{s} is said to be a \emph{proper} prefix of \emph{t}.

Simmetrically, a string \emph{s} is said to be a \textbf{suffix} of \emph{t} if there exists a string \emph{u} such that \emph{t = us}. If \emph{u} is nonempty, \emph{s} is said to be a \emph{proper} suffix of \emph{t}.

\subsubsection{Let us count: prefixes and suffixes}

The number of all posssible prefixes/suffixes of a string of length n is
\begin{center}
    \emph{prefixes/suffixes} $= n + 1$
\end{center}

\textbf{n are non-empty, +1} because we include also \textbf{the empty string} as prefix/suffix.

\subsection{Reverse and palindrome}

The \textbf{reverse} of a string is a string with the same symbols but in reverse order. For example, if \emph{s = abc} (where \emph{a}, \emph{b}, and \emph{c} are symbols of the alphabet), then the reverse of \emph{s} is \emph{cba}.

A string that is the reverse of itself is called a \textbf{palindrome}, which also includes the empty string and all strings of length 1.

\subsection{Rotations and permutations}

A string \emph{s = uv} is said to be a \textbf{rotation} of \emph{t} if \emph{t = vu}. For example, if $\Sigma = \{0, 1\}$ the string \emph{0011001} is a rotation of \emph{0100110}, where \emph{u = 00110} and \emph{v = 01}.

\subsubsection{Let us count: rotations}
Given a string S, with $|S| = n$, we denote \textbf{S=AB}, where \textbf{A is a prefix} substring, and \textbf{B a suffix} substring.

For every possible pair A, B the string \textbf{R=BA is a rotation of S}, including S itself, that is, the case where $A=\varepsilon$ and $B=S$.

Since we have n different non empty suffixes B, the number of rotations is
$$
    rotations = n
$$

It is not \emph{n + 1} because the two suffixes $B=S$ and $B=\varepsilon$ produce exactly the same rotation, given by S itself.

\subsubsection{Let us count: permutations}

Given a string S, with $|S| = n$, the number of permutations of the characters in S is
$$
    permutations = n!
$$

Notice that the formula does not consider the case of indentical permutations: that is, a string of \emph{n} identical characters, will have \emph{n} identical permutations.


\end{document}