\documentclass[../main.tex]{subfiles}

\begin{document}
\section{Alignment}

We have seen two symmetrical approaches to the problem of ``comparing sequences'': the edit distance measures how different two strings are, and the longest common subsequence, that measures how similar they are.
\textbf{What if we combine the two approaches into a unique one?}

\textbf{Idea:} given two strings S and T, define a similarity measure where
\begin{itemize}
\item \emph{differences} have a negative effect: define \textbf{negative scores} for the weight of \textbf{edit operations} (insertions, deletions, substitutions).
\item \emph{conserved letters} have a positive effect: define \textbf{positive scores} for the weight of \textbf{matches} between letters of the two strings.
\end{itemize}

The goal is now to \textbf{find the alignment that maximises the score} resulting by the weights just defined.

\subsection{Global alignment}

The goal of \textbf{global alignment} is to \textbf{try to align every character in every sequence}. We can find a solution for this problem once again using dynamic programming, in a similar way to the previous algorithms.

The rule for calculating the cell values is a combination of what we have seen so far:
$$
E(S(i), T(j)) = \bm{\max} \left\{\begin{array}{rcl}
E(S(i), T(j-1)) + \mathbf{w_d} \\ 
E(S(i-1), T(j)) + \mathbf{w_d} \\
E(S(i-1), T(j-1)) + \mathbf{w_m} & \mbox{if} & s_i = t_j \\
E(S(i-1), T(j-1)) + \mathbf{w_s} & \mbox{if} & s_i \neq t_j \\
\end{array}
\right.
$$
We are still looking for the ``max'': we're trying to find the alignment with the maximum score.
As before, we can use pointers to keep track of the choice made for each cell.

For the example we set $\mathbf{w_d=-2, w_s=-1, w_m=1}$.

\begin{center}
\begin{tabular}{|c|c|c|c|c|c|}
\hline
& \textbf{-} & \textbf{H} & \textbf{O} & \textbf{M} & \textbf{E} \\
\hline
\textbf{-} & \cellcolor[gray]{0.9}\tabel{0}{30} & \tabel{-2}{31} & \tabel{-4}{32} & \tabel {-6}{33} & \tabel{-8}{34} \\
\hline
\textbf{H} & \tabel{-2}{35} & \cellcolor[gray]{0.9}\tabel{1}{36} & \tabel{-1}{37} & \tabel{-3}{38} & \tabel{-5}{39} \\
\hline
\textbf{O} & \tabel{-4}{40} & \tabel{-1}{41} & \cellcolor[gray]{0.9}\tabel{2}{42} & \tabel{0}{43} & \tabel{-2}{44} \\
\hline
\textbf{U} & \tabel{-6}{45} & \tabel{-3}{46} & \tabel{0}{47} & \cellcolor[gray]{0.9}\tabel{1}{48} & \tabel{-1}{49} \\
\hline
\textbf{S} & \tabel{-8}{50} & \tabel{-5}{51} & \tabel{-2}{52} & \cellcolor[gray]{0.9}\tabel{-1}{53} & \tabel{0}{54} \\
\hline
\textbf{E} & \tabel{-10}{55} & \tabel{-7}{56} & \tabel{-4}{57} & \tabel{-3}{58} & \cellcolor[gray]{0.9}\tabel{0}{59} \\
\hline
\end{tabular}
\begin{tikzpicture}[overlay, remember picture, shorten >=.5pt, shorten <=.5pt, transform canvas={yshift=.25\baselineskip}, thick]
	\tabln{59}{53}
	\draw [->]({pic cs:53r}) to ({pic cs:48r});
	\tabln{48}{42}
	\tabln{42}{36}
	\tabln{36}{30}
\end{tikzpicture}
\end{center}

Backtracking as before we can build the alignment:
\begin{center}
\tt 
H O M - E \\
H O U S E
\end{center}

\subsubsection{Needleman-Wunsch algorithm}

The Needleman-Wunsch is the most famous global alignment algorithm and is  exactly the same as above but formalized in a different way.
$$
M(i, j) = \bm{\max} \left\{\begin{array}{rcl}
&M(i, j-1) + \mathbf{g} \\ 
&M(i-1, j) + \mathbf{g} \\
&M(i-1, j-1) + \mathbf{\bm{\sigma}(s_i, t_j)} \\
\end{array}
\right.
$$

\textbf{We no longer distinguish matches/mismatches}: the score of aligning $s_i$ and $t_j$ \textbf{is already included in the substitution matrix $\bm{\sigma(s_i, t_j)}$}

Given an alphabet $\bm{\Sigma}$\textbf{ the substitution matrix} is a $|\Sigma|\times|\Sigma|$ matrix that for every pair of characters $a_i, a_j \in \Sigma$ gives us the \textbf{``weight'' in the alignment of substituting $\mathbf{a_i}$ with $\mathbf{a_j}$}. It has the following properties:
\begin{itemize}
\item the matrix is symmetrical, i.e. $\sigma(a_i, a_j) = \sigma(a_j, a_i)$.
\item given $a_i$, the values $\sigma(a_i, a_j)$ for all the characters $a_j$ can be different in the matrix.
\item values for ``matches'' $\sigma(a_i, a_i)$ can be different for every letter $a_i$.
\end{itemize}

The simplest substitution matrices are the ones used for DNA/RNA sequences. The scores for matches (positive) and substitutions (negative) do not change with nucleotides. For example:

\begin{center}
\begin{tabular}{|c|c|c|c|c|}
\hline
& \textbf{A} & \textbf{C} & \textbf{G} & \textbf{T} \\
\hline
\textbf{A} & +2 & -1 & -1 & -1 \\
\hline
\textbf{C} & -1 & +2 & -1 & -1 \\
\hline
\textbf{G} & -1 & -1 & +2 & -1 \\
\hline
\textbf{T} & -1 & -1 & -1 & +2 \\
\hline
\end{tabular}
\end{center}

\subsection{Local alignment}

\end{document}
















